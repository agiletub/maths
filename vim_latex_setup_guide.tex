%Preamble%
\documentclass[11pt]{article}
\usepackage[utf8]{inputenc}
\usepackage[T1]{fontenc}

%Packages%
%TikZ%
\usepackage{physics}
\usepackage{amsmath}
\usepackage{tikz}
\usepackage{mathdots}
\usepackage{yhmath}
\usepackage{cancel}
\usepackage{color}
\usepackage{siunitx}
\usepackage{array}
\usepackage{multirow}
\usepackage{amssymb}
\usepackage{tabularx}
\usepackage{booktabs}
\usetikzlibrary{fadings}
\usetikzlibrary{patterns}
\usetikzlibrary{shadows.blur}
\usetikzlibrary{shapes}
\usepackage{extarrows}
%Document Layout
\usepackage{geometry} %Page dimensions
\usepackage{fancyhdr} %Custom Headers
\usepackage{footmisc} %Custom Footers
\usepackage{titlesec} %Title and Section Formatting
\usepackage{array}
%Formatting Packages%
\usepackage{xcolor} %For color%
\usepackage{graphicx} %For Images%
%\graphicspath{ {images/} }
%Miscellaneous Packages%
%Bibliographies
\usepackage{hyperref} %Hyperlinks
\usepackage{enumitem} %Advanced Lists
\usepackage{comment} %Multiline Comments
%Math Packages%
\usepackage{mathtools, amsmath, nccmath, amsfonts, amssymb, amsthm, thmtools}
%Page Geometry%
\geometry{
    a4paper,
    total={170mm,257mm},
    left=20mm,
    top=20mm,
    headheight=14pt
}


%%%%%%%%%%%Dark Mode Stuff%%%%%%%%%%%%%%
%\definecolor{mybgcolor}{RGB}{6.9, 6.9, 6.9}
%\pagecolor{mybgcolor}
%\color[rgb]{1,1,1}
%%%%%%%%%%%%%%%%%%%%%%%%%%%%%%%%%%%%%%%%

%Theorem Environments%
%Ref:https://math.ucsd.edu/~jeggers/latex/amsthdoc.pdf%
\theoremstyle{plain}% default 
\newtheorem{thm}{Theorem}[section] 
\newtheorem{lem}{Lemma}[thm] 
\newtheorem{prop}[lem]{Proposition} 
\newtheorem{cor}{Corollary}[thm]
\newtheorem*{claim}{Claim}
\newtheorem{case}{Case}
\theoremstyle{definition}
\newtheorem{defn}{Definition}[section] 
\newtheorem*{ques}{Question}
\newtheorem{conj}{Conjecture}[section]
\newtheorem{exmp}{Example}[section]
\newtheorem{exer}{Exercise}[section]
\theoremstyle{remark} 
\newtheorem*{rem}{Remark} 
\newtheorem*{note}{Note}

%Command Changes%
\renewcommand{\contentsname}{Table of Contents}
\renewenvironment{proof}{{\bfseries Proof.}}{ \\ \qed}

%New Custom Commands%
%Math Notations
\newcommand{\R}{{\mathbb R}}
\newcommand{\Z}{{\mathbb Z}}
\newcommand{\N}{{\mathbb N}}
\newcommand{\Q}{{\mathbb Q}}
\newcommand{\C}{{\mathbb C}}
\newcommand{\multdot}{\text{·}}
\newcommand{\sq}{\ensuremath{^2}}
\newcommand\numberthis{\addtocounter{equation}{1}\tag{\theequation}}
%Simple Footnote
\newcommand{\simplefn}[1]{\hfill \sl{\footnotesize{#1}} \\ }
%Section Decoration
\newcommand{\sectiondecor}{
    \begin{center}
    \vspace{-0.2in}
    \line(1,0){300} \\
    \large{$\infty$}    
    \end{center}
}
%Syllabus Commands
\newcommand{\infopointer}[2]{\large{\textbf{#1:}} \textit{#2}}
%For Notes:
\newcommand{\remarklinebreak}{\hfill \break \vspace{-0.3in}}

%%%MAIN DOC BEGINS%%%
\title{Vim + \LaTeX: Setup Guide for Arch Linux}
\author{\textbf{Sanjyot Shenoy}}
\date{2021}


\pagestyle{fancy}
\fancyhf{}
\lhead{Vim + \LaTeX}
\rhead{Sanjyot Shenoy}
\rfoot{Page \thepage}
\renewcommand{\footrulewidth}{0.4pt}

\begin{document}
	\maketitle
	\tableofcontents
	\section{Setting up of Arch Linux}
	First we begin with installation of the Arch Linux Distro which is the most popopular Distro, known for its simplicity.
	\begin{enumerate}
		\item First get the archlinux.iso from internet. Assuming you have a VMWare Fusion as your VM Provider, boot up the VM from this .iso file. 
		\item Now follow this guide on setting up ArchLinux:\\ \url{https://phoenixnap.com/kb/arch-linux-install}.
		\item \textbf{Note:} There appears to be some problem while setting up the locale \texttt{en\_IN.UTF-8} so it is recommeneded to use the locale  \texttt{en\_IN.UTF-8}, to check whether locale settings are correctly set up, use the \texttt{locale} command.
		\item After Arch has been set up, we will install the GNOME Desktop Environment with a Display Manager. For this we follow the guide given at:\\ \url{https://phoenixnap.com/kb/arch-linux-gnome}. 
		\item Note that, many basic packages need to be installed like \texttt{sudo, git}, etc., also it is recommended to make a new user which will be the main user by using \texttt{su - (your\_username)} and then giving it \texttt{sudo} privileges.	
	\end{enumerate}
	\section{Setting up Vim and Vim-Plug}
	Now we wish to set up Vim and its plugins through the vim-plug Plugin Manager, we could also use Vundle or others, but this is the most efficient plugin manager so we shall go with it. To make use of clipboard copy-paste in Vim, we install gVim (GUI Vim). Again, from here on, we will be a primary user(we will go with the name 'sanju' for simplicity) with sudo privileges and not root. 
	\begin{enumerate}
		\item First switch to the non-root user: \\
			\texttt{su - sanju}
		\item Now install gVim: \\
			\texttt{ sudo pacman -Ss gvim \\
			sudo pacman -S gvim}\\
		Note: \texttt{-Ss} is used to search packages, and \texttt{-S} is used for installation.
		\item Now install vim-plug using following: \\
			\texttt{
curl -fLo ~/.vim/autoload/plug.vim --create-dirs \textbackslash \\
    https://raw.githubusercontent.com/junegunn/vim-plug/master/plug.vim } \\
    or just by Downloading \url{https://raw.githubusercontent.com/junegunn/vim-plug/master/plug.vim} and putting it inside \texttt{~/.vim/autload}.
    		\item Verify vim-plug Install by starting vim and using the command: \\ 
			\texttt{:PlugInstall}
	\end{enumerate}
	If the command doesn't throw an error then vim-plug is correctly installed. Now we need to install \TeX-Live Package which is the actual \LaTeX package.
	\section{Installing \TeX-Live Package}
	Now we wish to install \LaTeX on to our system. For this, we just need to use the following command: \\
	\texttt{sudo pacman -S texlive-most}\\
	Depending on the internet speed, this will take some time. You can also add additional languages support to \TeX by using: \\
	\texttt{sudo pacman -S texlive-lang}\\
	and then specifying the option numbers of the packages you wish to install.
	\section{Installing and configuring vim plugins}
	\textbf{From now on, do NOT use vim as \texttt{sudo vim} as there are differences between the user and root vim configurations.} \\
	Before we can edit the Vim Configuration File which is the .vimrc file, we need to first grant the owenership of the configuration file to the non-root user by the command: \\
	\texttt{sudo chown sanju $\sim$/.vimrc}\\
	And now we can edit the configuration file without using \texttt{sudo}.\\
	Now, we edit the .vimrc file by the following command: \\
	\texttt{vim $\sim$/.vimrc}\\
	Now, edit your .vimrc file to make it look like this: \\
	\texttt{call plug\#begin() \\
	Plug 'sirver/ultisnips' \\
	    let g:UltiSnipsExpandTrigger = '<tab>' \\
	    let g:UltiSnipsJumpForwardTrigger = '<tab>' \\
	    let g:UltiSnipsJumpBackwardTrigger = '<s-tab>'\\
	    let g:UltiSnipsSnippetDir=[\$HOME.'/.vim/UltiSnips'] \\
	\\
	Plug 'lervag/vimtex' \\
	    let g:tex\_flavor='latex' \\
	    let g:vimtex\_view\_method='zathura'\\
	    let g:vimtex\_quickfix\_mode=0\\
	\\
	Plug 'preservim/nerdtree'\\
	\\
	Plug 'vim-airline/vim-airline'\\
	\\
	Plug 'KeitaNakamura/tex-conceal.vim'\\
	    set conceallevel=1\\
	    let g:tex\_conceal='abdmg'\\
	    hi Conceal ctermbg=none\\
	\\
	Plug 'dylanaraps/wal'\\
	colorscheme wal\\
	set background=dark\\
	\\
	call plug\#end()\\
	\\
	:set nu \\
	setlocal spell\\
	set spelllang=en\_us\\
	"inoremap <C-l> <c-g>u<Esc>[s1z=`]a<c-g>u}\\
	After setting the configuration file, we now install the plugins by: \\
	\texttt{vim \\
	:PlugInstall} \\
	Now, we wait for sometime and after the plugins are successfully installed, we now proceed to install our snippets.
	\section{Setting up snippets for \TeX}
	Now comes the most important part, for which we went through the entire ordeal above. We create the snippets file by the following command: \\
	\texttt{vim $\sim$/.vim/UltiSnips/tex.snippets}\\
	Now, set your desired snippets as you like. My tex.snippets file is available online. 
		
\end{document}
