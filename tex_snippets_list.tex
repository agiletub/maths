%Preamble%
\documentclass[11pt]{article}
\usepackage[utf8]{inputenc}
\usepackage[T1]{fontenc}

%Packages%
%TikZ%
\usepackage{physics}
\usepackage{amsmath}
\usepackage{tikz}
\usepackage{mathdots}
\usepackage{yhmath}
\usepackage{cancel}
\usepackage{color}
\usepackage{siunitx}
\usepackage{array}
\usepackage{multirow}
\usepackage{amssymb}
\usepackage{tabularx}
\usepackage{booktabs}
\usetikzlibrary{fadings}
\usetikzlibrary{patterns}
\usetikzlibrary{shadows.blur}
\usetikzlibrary{shapes}
\usepackage{extarrows}
\usepackage{longtable}
%Document Layout
\usepackage{geometry} %Page dimensions
\usepackage{fancyhdr} %Custom Headers
\usepackage{footmisc} %Custom Footers
\usepackage{titlesec} %Title and Section Formatting
\usepackage{array}
%Formatting Packages%
\usepackage{xcolor} %For color%
\usepackage{graphicx} %For Images%
%\graphicspath{ {images/} }
%Miscellaneous Packages%
%Bibliographies
\usepackage{hyperref} %Hyperlinks
\usepackage{enumitem} %Advanced Lists
\usepackage{comment} %Multiline Comments
%Math Packages%
\usepackage{mathtools, amsmath, nccmath, amsfonts, amssymb, amsthm, thmtools}
%Page Geometry%
\geometry{
    a4paper,
    total={170mm,257mm},
    left=20mm,
    top=20mm,
    headheight=14pt
}


%%%%%%%%%%%Dark Mode Stuff%%%%%%%%%%%%%%
%\definecolor{mybgcolor}{RGB}{6.9, 6.9, 6.9}
%\pagecolor{mybgcolor}
%\color[rgb]{1,1,1}
%%%%%%%%%%%%%%%%%%%%%%%%%%%%%%%%%%%%%%%%

%Theorem Environments%
%Ref:https://math.ucsd.edu/~jeggers/latex/amsthdoc.pdf%
\theoremstyle{plain}% default 
\newtheorem{thm}{Theorem}[section] 
\newtheorem{lem}{Lemma}[thm] 
\newtheorem{prop}[lem]{Proposition} 
\newtheorem{cor}{Corollary}[thm]
\newtheorem*{claim}{Claim}
\newtheorem{case}{Case}
\theoremstyle{definition}
\newtheorem*{ques}{Question}
\newtheorem{conj}{Conjecture}[section]
\newtheorem{exmp}{Example}[section]
\newtheorem{exer}{Exercise}[section]
\theoremstyle{remark} 
\newtheorem*{note}{Note}

\newtheoremstyle{breakthm}% %For leaving a line after theorem header
	{}{}%
	{\itshape}{}%
	{\bfseries}{}%
	{\newline}{}
	% The above doesnt work really well with enum lists, so for that we have to start the theorem as 
	% \begin{breakthm}[...]
	% \leavevmode \vspace{-\baselineskip}
	% \begin{enumerate}
	%Use breakenum snippet for this!
	%For more info on thereom styles: Check this: http://www.ams.org/arc/tex/amscls/amsthdoc.pdf

\theoremstyle{breakthm}
\newtheorem{bthm}{Theorem}[section]
\newtheorem{blem}{Lemma}[section]

\newtheoremstyle{breakdefn}% %For leaving a line after defn header
	{}{}%
	{}{}%
	{\bfseries}{:}%
	{\newline}{}

\theoremstyle{breakdefn}
\newtheorem{defn}{Definition}[section] 
\newtheorem*{bques}{Question}
\newtheorem{bconj}{Conjecture}[section]
\newtheorem{bexmp}{Example}[section]
\newtheorem{bexer}{Exercise}[section]

\newtheoremstyle{breakrem}% %For leaving a line after remark header
	{}{}%
	{}{}%
	{\itshape \bfseries}{:}%
	{\newline}{}
	
\theoremstyle{breakrem}
\newtheorem*{rem}{Remarks}

%Command Changes%
\renewcommand{\contentsname}{Table of Contents}
\renewenvironment{proof}{{\bfseries Proof.}}{ \\ \qed}

%New Custom Commands%
%Math Notations
\newcommand{\R}{{\mathbb R}}
\newcommand{\Z}{{\mathbb Z}}
\newcommand{\N}{{\mathbb N}}
\newcommand{\Q}{{\mathbb Q}}
\newcommand{\C}{{\mathbb C}}
\newcommand{\multdot}{\text{·}}
\newcommand{\sq}{\ensuremath{^2}}
\newcommand\numberthis{\addtocounter{equation}{1}\tag{\theequation}}
%Simple Footnote
\newcommand{\simplefn}[1]{\hfill \sl{\footnotesize{#1}} \\ }
%Section Decoration
\newcommand{\sectiondecor}{
    \begin{center}
    \vspace{-0.2in}
    \line(1,0){300} \\
    \large{$\infty$}    
    \end{center}
}
%Syllabus Commands
\newcommand{\infopointer}[2]{\large{\textbf{#1:}} \textit{#2}}
%For Notes:
\newcommand{\remarklinebreak}{\hfill \break \vspace{-0.3in}}

%%%MAIN DOC BEGINS%%%
\title{\TeX\, Snippets List}
\author{\textbf{Sanjyot Shenoy}}
\date{Last Update: 22/05/2021}


\pagestyle{fancy}
\fancyhf{}
\lhead{\TeX\, Snippets List}
\rhead{Sanjyot Shenoy}
\rfoot{Page \thepage}
\renewcommand{\footrulewidth}{0.4pt}

\begin{document}
	\maketitle
	\tableofcontents
	\section{Introduction}
	This resource is like a reference for all the snippets. If you wish to understand how the snippets work and some syntax related help, you should check out he UltiSnips: Snippets guide. \par
	So the entire list is represented as a table, with columns explaining a certain aspect of the snippets.
	\section{Table Legend}
	\begin{enumerate}
		\item\textbf{Trigger Word}: The trigger word is what needs to be typed in order to trigger the corresponding snippet. The snippet may be triggered automatically or by using the <Tab> trigger key.
		\item\textbf{Description}: The description of the snippet.
		\item\textbf{Priority}: Priority number of the snippet.
		\item\textbf{Snippet Options}: The snippet options are the ones UltiSnips uses to understand how to and when to trigger the snippet, for more information, you can look up the snippet guide document
		\item\textbf{Math Mode Only?}: Some snippets can be triggered only inside math mode and this is to distinguish those.
		\item\textbf{Notes}: Some tips or remarks on the snippet.
	\end{enumerate}
	\section{Snippet Reference List}
	\subsection{General \LaTeX\ Command Snippets}
	\begin{center}
	\begin{table}[htbp]
		\begin{tabular}{||m{2cm}|m{1.5cm}|m{3cm}|m{1.5cm}|m{2.5cm}|m{4cm}||}
		\hline
		\textbf{Trigger Word} & \textbf{Priority} & \textbf{Description} & \textbf{Options} & \textbf{Math Mode Only?} & \textbf{Notes} \\
		\hline
		\hline
		template & 0 &Basic Template Generation & b & N & Use tab triggers to generate title, date and header text. \\ \hline
		beg & 0 & Begin Environment & bA & N & Use tab tigger to move inside the block with indent. \\ \hline
		... & 100 & \textbackslash ldots & iA & N & Lower Dots \\ \hline 
		\end{tabular}
	\end{table}
	\newpage
	\begin{table}[htbp]
		\begin{tabular}{||m{2cm}|m{1.5cm}|m{3cm}|m{1.5cm}|m{2.5cm}|m{4cm}||}
		\hline
		\textbf{Trigger Word} & \textbf{Priority} & \textbf{Description} & \textbf{Options} & \textbf{Math Mode Only?} & \textbf{Notes} \\
		\hline
		\hline
		enum & 0 & Enumerate List & bA & N &  \\ \hline
		breakenum & 0 & Break Thm. Enum. List & bA & N &  \\ \hline
		item & 0 & Item List & bA & N &  \\ \hline
		breakitem & Break Thm. Item List & bA &  & N &  \\ \hline
		desc & 0 & Description List & b & N &  \\ \hline
		breakdesc & 0 & Break Thm. Desc. List & b & N &  \\ \hline
		pac & 0 & Package & b & N & Use pacakage command \\ \hline
		mk & Inline Maths & 0 & wA & N &  \\ \hline
		dm & Display Maths & 0 & wA & N &  \\ \hline
		ali & Align Maths & 0 & bA & N &  \\ \hline
		 &  & 0 &  &  &  \\ \hline
		 &  & 0 &  &  &  \\ \hline
		 &  & 0 &  &  &  \\ \hline
		 &  & 0 &  &  &  \\ \hline
		 &  & 0 &  &  &  \\ \hline
		 &  & 0 &  &  &  \\ \hline
		 &  & 0 &  &  &  \\ \hline
		\end{tabular}
	\end{table}
	\end{center}
	\subsection{Maths Notation Snippets}	
	\begin{center}
	\begin{table}[htbp]
		\begin{tabular}{||m{2cm}|m{1.5cm}|m{3cm}|m{1.5cm}|m{2.5cm}|m{4cm}||}
		\hline
		\textbf{Trigger Word} & \textbf{Priority} & \textbf{Description} & \textbf{Options} & \textbf{Math Mode Only?} & \textbf{Notes} \\
		\hline
		\hline
		... & 100 & \textbackslash ldots & iA & N & Lower Dots \\ \hline 	
		=> &  & 0 & Ai &  &  \\ \hline
		=< &  & 0 & Ai &  &  \\ \hline
		iff &  & 0 &  &  &  \\ \hline
		 &  & 0 &  &  &  \\ \hline
	\end{tabular}
	\end{table}
	\end{center}
\end{document}
